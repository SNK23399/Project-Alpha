\section{Conclusion}
\label{sec:conclusion}

This document has presented a systematic approach to core-satellite portfolio management using signal-based ETF selection. Through comprehensive feature engineering, walk-forward backtesting, and rigorous statistical testing, we have identified an effective and robust strategy for generating consistent alpha.

\subsection{Key Findings}

\textbf{1. IC-weighted feature ensembles generate statistically significant alpha.}

The Information Coefficient (IC) weighting mechanism is the only tested improvement that achieves statistical significance at the 5\% level (p = 0.047). By weighting features according to their rolling correlation with forward alpha, the ensemble adapts to changing market conditions while maintaining robustness.

\textbf{2. Simplicity outperforms complexity.}

Despite testing numerous enhancements---including dynamic satellite count, stability weighting, time-series features, and regime-adaptive signals---none achieved statistical significance beyond the basic IC-weighted approach. This supports the principle that simpler strategies are often more robust out-of-sample.

\textbf{3. N=5 satellites provides optimal balance.}

Analysis across $N \in \{1, \ldots, 10\}$ reveals that N=5 maximizes the hit rate (92.8\%) while maintaining meaningful alpha (+4.88\% annualized). Lower N increases alpha but reduces consistency; higher N provides stability at the cost of alpha dilution.

\textbf{4. Results are temporally robust.}

The strategy outperforms the baseline in all 10 years tested, with consistent alpha across early, middle, and late sub-periods. Rolling alpha is positive 87.4\% of the time. This rules out early-luck compounding as the source of returns.

\subsection{Final Strategy Specification}

Based on our analysis, the recommended strategy is:

\begin{table}[h]
\centering
\caption{Final Strategy Parameters}
\begin{tabular}{ll}
\toprule
\textbf{Parameter} & \textbf{Value} \\
\midrule
Core allocation & 60\% (iShares MSCI ACWI) \\
Satellite allocation & 40\% (5 ETFs at 8\% each) \\
Selection method & IC-weighted feature ensemble \\
Number of satellites & Fixed N=5 \\
Rebalancing frequency & Monthly \\
Holding period & 1 month \\
\bottomrule
\end{tabular}
\end{table}

\textbf{Expected performance:}
\begin{itemize}
    \item Annualized alpha: approximately +4.9\% over ACWI
    \item Monthly hit rate: approximately 93\%
    \item Maximum drawdown: approximately $-7\%$ relative to benchmark
    \item Sharpe ratio: approximately 1.07
\end{itemize}

\subsection{Practical Implications}

For long-term investors with a 30+ year horizon, this strategy offers several advantages:

\textbf{Systematic execution:} Clear rules eliminate emotional decision-making and ensure consistent implementation.

\textbf{Tax efficiency:} Irish-domiciled accumulating ETFs minimize dividend taxes. Infrequent trading (monthly) reduces capital gains events.

\textbf{Low costs:} With EUR 1--3 per trade and low-expense ETFs (TER $<$ 0.30\%), transaction costs do not significantly erode alpha.

\textbf{Scalability:} The strategy works for portfolios from EUR 10,000 to EUR 1,000,000+ without modification.

\textbf{Profit locking:} Monthly rebalancing automatically transfers satellite gains into the stable core, reducing portfolio risk over time.

\subsection{Limitations and Future Work}

While the strategy demonstrates robust historical performance, several areas warrant further investigation:

\textbf{Extended out-of-sample testing:} Prospective implementation would provide the strongest evidence of strategy validity.

\textbf{Alternative asset classes:} The signal framework could be extended to bonds, commodities, or factor ETFs to further diversify alpha sources.

\textbf{Machine learning integration:} Non-linear models might capture additional predictive relationships, though care must be taken to avoid overfitting.

\textbf{Transaction cost optimization:} More sophisticated rebalancing rules could reduce turnover while maintaining signal freshness.

\subsection{Closing Remarks}

The core-satellite framework, combined with systematic signal-based selection, offers a compelling approach to long-term wealth accumulation. By prioritizing statistical rigor over backtest optimization, we have identified a strategy that balances alpha generation with robustness and simplicity.

The key insight from this work is that \textbf{statistical significance matters more than backtest performance}. Many apparent improvements fail to achieve significance when properly tested. By accepting only IC weighting---the sole statistically significant enhancement---we reduce the risk of implementing strategies that will not persist out-of-sample.

For investors seeking consistent outperformance over a multi-decade horizon, the IC-weighted core-satellite strategy provides a principled, evidence-based approach to portfolio construction.
