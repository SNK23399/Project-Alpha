\section{Methodology}
\label{sec:methodology}

This section presents our systematic approach for ETF selection in a core-satellite portfolio framework. We develop a quantitative signal-based methodology that identifies satellite ETFs with the highest probability of generating positive alpha relative to a global market benchmark.

\subsection{Portfolio Framework}
\label{subsec:portfolio-framework}

We adopt a core-satellite portfolio structure that combines the stability of passive global diversification with the alpha-generation potential of tactical satellite positions. The core allocation consists of a global market-cap weighted index (iShares MSCI ACWI), representing approximately 3,000 stocks across developed and emerging markets. This provides broad exposure while serving as our benchmark for measuring excess returns.

The satellite allocation dynamically selects from a filtered universe of equity ETFs based on predictive signals. We allocate up to 40\% to satellites, distributed equally among a maximum of four selected ETFs. When no satellites demonstrate sufficient alpha potential, the strategy defaults to 100\% core allocation, ensuring we never force suboptimal positions.

\subsection{ETF Universe Construction}
\label{subsec:universe}

Starting from a catalog of over 7,500 ETFs, we apply a series of practical filters to construct our investment universe. We restrict to Irish-domiciled ETFs for tax efficiency, require accumulating distribution policies to avoid dividend drag, and impose a maximum total expense ratio of 0.30\%. To ensure adequate liquidity and data availability, we require a minimum fund size of 100 million EUR and at least five years of price history. These filters yield approximately 500 eligible ETFs that form our selection universe.

\subsection{Signal Base Computation}
\label{subsec:signal-bases}

The foundation of our selection methodology lies in computing a comprehensive set of 167 signal bases from raw price data. We organize these into 27 categories, each capturing different aspects of ETF behavior relative to the core benchmark. Most signals use day-based window parameters (21d, 63d, 126d, 252d) corresponding approximately to 1, 3, 6, and 12 months respectively, with trend indicators preserving industry-standard windows (e.g., 50-day, 200-day moving averages).

\subsubsection{Return and Alpha Signals}

The most fundamental signals derive from daily returns $r_{i,t} = (P_{i,t} - P_{i,t-1})/P_{i,t-1}$ and excess returns (alpha) $\alpha_{i,t} = r_{i,t} - r_{c,t}$, where $r_{c,t}$ denotes the core benchmark return. We also compute price ratios $P_{i,t}/P_{c,t}$ and their logarithmic transformations to capture relative valuation dynamics.

\subsubsection{Relative Strength}

To measure momentum relative to the benchmark, we compute multi-horizon relative strength as the ratio of ETF momentum to core momentum:
\begin{equation}
    \text{RS}_{i,t}^{(n)} = \frac{(P_{i,t} - P_{i,t-n}) / P_{i,t-n}}{(P_{c,t} - P_{c,t-n}) / P_{c,t-n}}
\end{equation}
We evaluate lookback periods of 21, 63, 126, and 252 trading days, corresponding approximately to 1, 3, 6, and 12 months. Additionally, we compute beta-adjusted relative strength by normalizing RS by the ETF's rolling beta:
\begin{equation}
    \text{BetaAdjRS}_{i,t} = \frac{\text{RS}_{i,t}^{(252)}}{|\beta_{i,t}| + \delta}
\end{equation}
where $\delta \in \{0.3, 0.5, 1.0\}$ is a damping parameter to prevent extreme values for low-beta ETFs.

\subsubsection{Risk Metrics}

Understanding risk characteristics is essential for satellite selection. We compute rolling beta as the covariance of ETF returns with core returns divided by core variance, using 21, 63, and 126-day windows. Idiosyncratic return adjusts alpha by residual volatility. Rolling volatility (annualized standard deviation of returns) captures absolute risk at 21 and 63-day horizons, while relative volatility normalizes this against the core's volatility. Drawdown, measured as the percentage decline from the running maximum price, quantifies downside exposure. We also compute relative drawdown as the difference between ETF and core drawdowns.

\subsubsection{Risk-Adjusted Performance}

Raw returns can be misleading without risk context. We compute rolling Sharpe ratios (return per unit volatility), Sortino ratios (return per unit downside deviation), and Information ratios (alpha per unit tracking error) over 126-day windows. These metrics identify ETFs delivering efficient risk-adjusted returns rather than merely high absolute returns.

\subsubsection{Trend Indicators}

Price trends relative to moving averages provide insight into momentum regimes. We measure the percentage deviation of price from its 50-day and 200-day moving averages, as well as the ``golden cross'' ratio of the 50-day to 200-day moving average. These signals help distinguish between trending and mean-reverting market conditions.

\subsubsection{Mean Reversion Signals}

A key innovation in our framework is the comprehensive treatment of mean reversion signals, which identify potentially oversold conditions that may precede recovery. We compute price z-scores at 63, 126, and 252-day windows (deviation from rolling mean normalized by rolling standard deviation), inverted such that oversold conditions yield positive values. The Bollinger Band position measures where price sits within its volatility bands. Classical oscillators including the Relative Strength Index and Stochastic oscillator are inverted to generate buy signals when assets appear oversold.

We also compute alpha z-scores, RS z-scores at 126 and 252-day windows, distance from 20 and 100-day moving averages, and the price ratio z-score. Perhaps most importantly, we introduce the alpha drawdown reversion signal:
\begin{equation}
    \text{AlphaDD}_{i,t} = -\left( \sum_{s=t-126}^{t} \alpha_{i,s} - \max_{u \leq t}\left(\sum_{s=u-126}^{u} \alpha_{i,s}\right) \right)
\end{equation}
This captures how far an ETF's cumulative alpha has fallen from its peak, identifying assets that have significantly underperformed recently and may be poised for mean reversion.

\subsubsection{Higher Moments}

To capture tail risk characteristics, we compute 63-day rolling skewness and kurtosis of returns. Negative skewness indicates asymmetric downside risk, while high kurtosis signals fat tails and potential for extreme moves.

\subsubsection{Correlation and Dispersion}

We measure each ETF's 63-day rolling correlation with the core benchmark, as well as an approximation of average pairwise correlation with the broader universe (computed efficiently as correlation with the equal-weighted market average). Low correlation signals diversification potential, while high correlation indicates the ETF moves in lockstep with the market.

\subsubsection{Momentum Dynamics}

Beyond static momentum, we capture acceleration and deceleration in relative strength. The momentum acceleration signals measure the 21-day and 63-day change in 6-month or 12-month relative strength, identifying ETFs whose outperformance is increasing or decreasing.

\subsubsection{Signal Disagreement Features}

We compute disagreement signals between short-term and long-term momentum. The bullish disagreement signal $(\text{RS}_{12m} - \text{RS}_{1m})$ identifies long-term winners experiencing short-term pullbacks---potential buy opportunities. Conversely, bearish disagreement $(\text{RS}_{1m} - \text{RS}_{12m})$ flags short-term strength in long-term laggards. We also measure trend-momentum divergence by comparing price-vs-MA z-scores with RS z-scores.

\subsubsection{Regime-Adaptive Signals}

Market conditions affect which signals work best. We compute a volatility regime indicator (core volatility z-score relative to its 252-day history) and a trend regime indicator (core price vs 200-day MA). Two composite signals boost base signals based on regime:
\begin{align}
    \text{TrendBoostedMom}_{i,t} &= \text{RS}^{(126)}_{i,t} \times (1 + \max(0, \text{TrendRegime}_t)) \\
    \text{VolBoostedReversion}_{i,t} &= \text{AlphaDD}_{i,t} \times (1 + \min(2, \max(0, \text{VolRegime}_t)))
\end{align}
These signals amplify momentum in trending markets and mean reversion in high-volatility regimes.

\subsection{Causal Smoothing Filters}
\label{subsec:filters}

Raw signals contain substantial noise that can lead to spurious rankings. To address this while preserving causality (avoiding look-ahead bias), we apply 27 smoothing filters to each signal base. All filters are strictly causal---they use only past data points at each time step. We employ both standard windows (21 and 63 days) and specialized adaptive filters to capture different temporal dynamics.

\textbf{Raw (no filter):} The unfiltered signal serves as a baseline (1 filter).

\textbf{Exponential Moving Average (EMA):} We apply EMAs with 21 and 63-day spans, implemented using IIR filtering for computational efficiency (2 filters):
\begin{equation}
    \text{EMA}_t = \alpha \cdot x_t + (1-\alpha) \cdot \text{EMA}_{t-1}, \quad \alpha = \frac{2}{n+1}
\end{equation}

\textbf{Double EMA (DEMA):} Reduced-lag smoothing computed as $\text{DEMA} = 2 \cdot \text{EMA} - \text{EMA}(\text{EMA})$ with 21 and 63-day spans (2 filters).

\textbf{Triple EMA (TEMA):} Minimal-lag smoothing using $\text{TEMA} = 3 \cdot \text{EMA} - 3 \cdot \text{EMA}^2 + \text{EMA}^3$ with 21 and 63-day spans (2 filters).

\textbf{Zero-Lag EMA (ZLEMA):} Momentum-adjusted EMA that compensates for lag by using price momentum, with 21 and 63-day spans (2 filters).

\textbf{Hull Moving Average:} The Hull MA combines weighted moving averages to reduce lag while maintaining smoothness, using 21 and 63-day periods (2 filters):
\begin{equation}
    \text{Hull}_t = \text{WMA}\left(2 \cdot \text{WMA}(x, n/2) - \text{WMA}(x, n), \sqrt{n}\right)
\end{equation}

\textbf{Triangular MA (TRIMA):} Double-smoothed SMA providing center-weighted averaging, with 21 and 63-day windows (2 filters).

\textbf{Gaussian MA:} Bell-curve weighted averaging with smooth frequency rolloff, using 21 and 63-day windows (2 filters).

\textbf{Butterworth Low-Pass Filter:} Second-order Butterworth filters with cutoff periods of 21 and 63 days provide excellent noise reduction with a smooth frequency response (2 filters). We use SciPy's \emph{lfilter} function for causal (one-directional) filtering.

\textbf{Kalman Filter:} An adaptive smoothing filter that adjusts based on signal-to-noise ratio (2 filters). We use two configurations: ``fast'' (process variance $10^{-4}$) for responsive tracking and ``slow'' (process variance $10^{-5}$) for stronger smoothing.

\textbf{Savitzky-Golay Filter:} Polynomial smoothing that preserves local maxima and minima better than moving averages, implemented with 21 and 63-day windows using cubic polynomials (2 filters). We ensure causality by setting the filter origin to use only past points.

\textbf{Median Filter:} Outlier-robust filtering that preserves edges, with 21 and 63-day windows (2 filters).

\textbf{Kaufman Adaptive MA (KAMA):} Adapts smoothing based on market efficiency ratio, distinguishing trending from choppy markets, using a 21-day period (1 filter).

\textbf{Regime-Switching Filter:} Automatically switches between fast (10-day) and slow (50-day) smoothing based on market volatility regime (1 filter).

Combining 167 signal bases with 27 filters yields 4,509 signal variants. Each filter-signal combination captures different temporal dynamics, allowing the selection algorithm to identify which combinations provide the strongest predictive power.

\subsection{Indicator Transformations}
\label{subsec:indicators}

Each filtered signal variant undergoes 25 indicator transformations that capture different aspects of the signal's behavior. These indicators form the final predictive features used for ETF selection.

\textbf{Level:} The raw filtered signal value.

\textbf{Momentum indicators:} We compute the change in signal value over 5, 21, 63, and 126-day horizons, capturing short to medium-term signal dynamics.

\textbf{Z-score indicators:} Signal deviation from its rolling mean, normalized by rolling standard deviation, computed at 21, 63, and 126-day windows. These identify when signals are at extreme values relative to recent history.

\textbf{Cross-sectional indicators:} At each time point, we compute the cross-sectional rank (ETF's percentile rank among all ETFs, yielding a uniform distribution in $[0,1]$) and cross-sectional z-score (deviation from cross-sectional mean divided by cross-sectional standard deviation). These ensure selection is based on relative positioning rather than absolute magnitudes.

\textbf{Trend indicators:} Ratios of short to long-term moving averages of the signal---MA(21)/MA(63), MA(63)/MA(252), and a longer-term variant---identify whether the signal is trending up or down.

\textbf{Mean reversion indicators:} Distance from 63, 126, and 252-day rolling means, identifying when signals have deviated significantly and may revert.

\textbf{Velocity and acceleration:} First and second derivatives of the signal (normalized by rolling standard deviation), capturing the rate and change in rate of signal movement.

\textbf{Drawdown:} Distance from the 63-day rolling maximum, measuring how far the signal has fallen from recent peaks.

\textbf{Range position:} Where the signal sits within its 21-day high-low range, normalized to $[0,1]$.

\textbf{Volatility ratios:} Ratio of short-term (5-day) to longer-term (21, 63, 126-day) signal volatility, identifying periods of unusual signal movement.

\textbf{Percentile:} The signal's position within its 252-day rolling distribution.

\textbf{Relative volatility:} Signal volatility normalized by core benchmark signal volatility.

Combining 4,509 signal variants with 25 indicators yields 112,725 total features. This comprehensive feature space allows systematic identification of which signal-filter-indicator combinations have genuine predictive power.

\subsection{Cross-Sectional Normalization}
\label{subsec:cross-sectional}

Beyond the cross-sectional indicators described above, we apply normalization to ensure comparability across features. Signal values vary dramatically in scale and distribution across different signal types and market regimes. The cross-sectional rank divides each ETF's rank by the count of available ETFs, yielding a uniform distribution between 0 and 1. The cross-sectional z-score subtracts the cross-sectional mean and divides by cross-sectional standard deviation. These transformations ensure that selection decisions are based on relative positioning within the current universe rather than absolute signal magnitudes.

\subsection{Portfolio Strategy Implementation}
\label{subsec:strategy}

Our implementation translates the feature framework into an actionable portfolio strategy through a systematic quarterly rebalancing process.

\subsubsection{Target Variable: 3-Month Forward Alpha}

The fundamental prediction task is identifying ETFs that will generate positive alpha over the next 3 months (63 trading days). For each ETF $i$ at time $t$, we define:
\begin{equation}
    \alpha_{i,t+63}^{(fwd)} = \frac{P_{i,t+63}}{P_{i,t}} - \frac{P_{c,t+63}}{P_{c,t}}
\end{equation}
where $P_{i,t}$ is the ETF price and $P_{c,t}$ is the core benchmark (ACWI) price. This measures excess return over the benchmark during the holding period.

The 3-month horizon aligns with several considerations: it is long enough to allow technical signals to materialize while being short enough to adapt to changing market conditions; it corresponds to quarterly rebalancing frequency, minimizing transaction costs; and academic research suggests 3-6 month horizons are optimal for momentum and technical strategies \cite{jegadeesh1993returns}.

\subsubsection{Portfolio Construction}

The portfolio maintains a fixed core-satellite allocation:
\begin{itemize}
    \item \textbf{Core (60\%):} Constant allocation to iShares Core MSCI World (ISIN: IE00B4L5Y983)
    \item \textbf{Satellites (40\%):} Distributed equally across $N \in \{3, 4, 5\}$ selected ETFs
\end{itemize}

At each quarterly rebalancing date, we:
\begin{enumerate}
    \item Compute all features for all ETFs using data up to (but not including) the rebalancing date
    \item Rank ETFs by the selected feature(s)
    \item Select the top $N$ ETFs (or bottom $N$ for mean reversion strategies)
    \item Execute full rebalancing: sell all current satellites, adjust core to 60\% of total portfolio value, distribute 40\% equally across new satellites
    \item Hold positions for 3 months until next rebalancing
\end{enumerate}

\subsubsection{Profit Locking Mechanism}

A key innovation is the automatic risk-reduction through profit locking. When satellites outperform, their portfolio weight naturally increases beyond the 40\% target. Quarterly rebalancing mechanically sells this excess to restore the 60/40 ratio, thereby transferring gains from tactical positions into the stable core.

Example: Starting with EUR 50,000 (EUR 30,000 core, EUR 20,000 satellites), suppose satellites gain 20\% while core gains 5\% over 3 months. The portfolio grows to EUR 31,500 core + EUR 24,000 satellites = EUR 55,500 total. Rebalancing targets 60\% = EUR 33,300 core and 40\% = EUR 22,200 satellites, requiring EUR 1,800 in satellite profits to be locked into the core.

This creates a convex payoff structure: gains are systematically captured in low-risk assets, while losses trigger increased satellite allocation at lower valuations (potentially beneficial for mean reversion).

\subsubsection{Monthly Contributions}

Between quarterly rebalances, monthly contributions of EUR 1,000 are split 60/40 and added to existing positions proportionally. No trading occurs during these months—contributions simply dollar-cost-average into current holdings. This reduces timing risk and transaction costs while maintaining the target allocation.

\subsection{Feature Discovery and Selection}
\label{subsec:selection}

Given the large feature space (112,725 features), systematic evaluation is essential. We compute predictive metrics on a walk-forward quarterly basis to identify features with genuine alpha-generation capability.

\subsubsection{Information Coefficient (IC)}

The primary metric is the Information Coefficient, defined as the Spearman rank correlation between feature values and subsequent 3-month alpha:
\begin{equation}
    IC_t = \text{corr}_{\text{Spearman}}(\{f_{i,t}\}_{i=1}^{N_{ETF}}, \{\alpha_{i,t+63}^{(fwd)}\}_{i=1}^{N_{ETF}})
\end{equation}

IC directly measures a feature's ability to rank ETFs by future performance. Positive IC indicates momentum (higher feature values → higher future alpha), while negative IC indicates mean reversion (lower feature values → higher future alpha).

We compute IC at each quarterly rebalancing date over the full history, yielding a time series $\{IC_t\}$. The mean IC indicates average predictive power:
\begin{equation}
    \overline{IC} = \frac{1}{T}\sum_{t=1}^{T} IC_t
\end{equation}

The IC Information Ratio measures consistency:
\begin{equation}
    IR = \frac{\overline{IC}}{\text{std}(IC)}
\end{equation}

High IR indicates stable predictive power across different market regimes. Typically, $\overline{IC} > 0.05$ is considered meaningful, and $IR > 1.0$ suggests consistency.

\subsubsection{Selection Performance Metrics}

Beyond correlation, we evaluate actual selection outcomes:

\textbf{Average alpha:} The mean 3-month alpha achieved by selecting the top $N$ ETFs by feature value:
\begin{equation}
    \overline{\alpha}_{\text{top-N}} = \frac{1}{T}\sum_{t=1}^{T}\left(\frac{1}{N}\sum_{i \in \text{Top-N}_t} \alpha_{i,t+63}^{(fwd)}\right)
\end{equation}

This directly measures the strategy's return generation.

\textbf{Hit rate:} The fraction of quarters where top-$N$ selection generates positive alpha:
\begin{equation}
    \text{Hit Rate} = \frac{1}{T}\sum_{t=1}^{T} \mathbb{1}\left[\frac{1}{N}\sum_{i \in \text{Top-N}_t} \alpha_{i,t+63}^{(fwd)} > 0\right]
\end{equation}

High hit rates ($>60\%$) indicate consistent rather than sporadic outperformance.

\textbf{Quintile spread:} The difference in average alpha between top 20\% and bottom 20\% of ranked ETFs:
\begin{equation}
    \text{Spread}_t = \overline{\alpha}_{Q5,t} - \overline{\alpha}_{Q1,t}
\end{equation}

Large spreads indicate strong discriminative power across the entire distribution.

\subsubsection{Momentum vs. Mean Reversion}

A critical strategic decision is whether to follow momentum (select high feature values) or mean reversion (select low feature values). We test both:

\textbf{Momentum strategy:} Select ETFs with highest feature ranks (top $N$)

\textbf{Mean reversion strategy:} Select ETFs with lowest feature ranks (bottom $N$)

For each feature, we compute $\overline{\alpha}_{\text{top-N}}$ and $\overline{\alpha}_{\text{bottom-N}}$ separately. The superior strategy determines how we interpret the feature. This allows the data to reveal whether signals are more effective as momentum or contrarian indicators.

\subsubsection{Feature Ranking and Selection}

Features are ranked by their average alpha generation ($\overline{\alpha}_{\text{top-N}}$ or $\overline{\alpha}_{\text{bottom-N}}$, whichever is higher), with secondary criteria:
\begin{itemize}
    \item IC Information Ratio $> 1.0$ (consistency requirement)
    \item Hit rate $> 55\%$ (better than random)
    \item Statistically significant IC (t-statistic $> 2.0$)
\end{itemize}

The top-performing features are candidates for portfolio implementation. We also analyze feature redundancy by computing pairwise correlations among top features—highly correlated features provide similar information and need not be combined.

\subsection{Walk-Forward Backtesting}
\label{subsec:backtesting}

To obtain realistic performance estimates, we employ strict walk-forward backtesting. On each monthly rebalancing date, signals are computed using only data available up to that point. The selection algorithm ranks ETFs and chooses satellites based solely on historical information. Performance is then measured over the subsequent month on data unseen during signal computation.

This methodology prevents look-ahead bias and data snooping, ensuring that reported performance reflects what an investor could have actually achieved in real-time. We use a 252-day (one year) lookback for signal computation, monthly rebalancing on the last trading day of each month, and select $k=4$ satellite ETFs per rebalancing.

All filters are strictly causal---the Savitzky-Golay filter, for example, uses the \emph{origin} parameter to ensure only past points contribute to each output. This attention to causality is critical; non-causal filters can show artificially inflated performance by implicitly using future information.

\subsection{Performance Evaluation}
\label{subsec:metrics}

We evaluate feature quality through multiple complementary metrics:

\begin{itemize}
    \item \textbf{Top-$k$ advantage:} Monthly alpha of selected ETFs vs universe average
    \item \textbf{Hit rate:} Percentage of months with positive selection alpha
    \item \textbf{Quintile spread:} Return spread between top and bottom quintiles
\end{itemize}

We also analyze performance by filter type and indicator type to identify systematic patterns. For example, we compare the average top-$k$ advantage of Hull-filtered signals versus raw signals, or level indicators versus momentum indicators.

Critically, our primary optimization objective is maximizing the percentage of months with positive rolling alpha rather than maximizing cumulative alpha. This emphasis on consistency over magnitude reflects the practical reality that investors find steady outperformance more valuable than volatile returns that average to the same level.

