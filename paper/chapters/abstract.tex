\begin{abstract}
This document presents a systematic core-satellite portfolio strategy designed for long-term wealth accumulation through dynamic ETF selection. The strategy combines the stability of a passive global core (60\% allocation to MSCI ACWI) with tactical satellite positions (40\% allocation across 5 selected ETFs) that aim to generate consistent positive alpha. We develop a comprehensive signal-based framework consisting of 167 signal bases, 27 causal smoothing filters, and 25 cross-sectional indicators, yielding over 112,000 potential features for predicting 1-month forward alpha. Through rigorous walk-forward backtesting over 120+ months (2015--2025), we evaluate multiple strategy improvements and apply statistical significance testing to identify genuine alpha sources. Our key finding is that Information Coefficient (IC) weighted feature ensembles provide statistically significant improvement over the baseline (p = 0.047), generating approximately +4.9\% annualized alpha with a 92.8\% hit rate. Other proposed improvements---including dynamic N selection, stability weighting, and time-series features---fail to achieve statistical significance after proper testing. The strategy employs monthly rebalancing with a fixed N=5 satellites, selected through IC-weighted ensemble scoring of top-performing features. We demonstrate robustness through sub-period analysis, showing consistent outperformance across early, middle, and late periods of the backtest, ruling out early-luck compounding as the source of returns.
\end{abstract}

\begin{IEEEkeywords}
Core-satellite portfolio, ETF selection, alpha generation, signal-based strategy, momentum, Information Coefficient, walk-forward backtesting, statistical significance
\end{IEEEkeywords}
