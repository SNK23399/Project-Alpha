\section{Introduction}
\label{sec:introduction}

\subsection{Motivation}

Long-term wealth accumulation through equity investing faces a fundamental trade-off: passive index investing provides broad diversification and low costs but limits alpha generation potential, while active management promises outperformance but introduces higher fees, behavioral biases, and tracking error. The core-satellite portfolio framework offers a middle ground, combining the stability of passive core holdings with the upside potential of tactical satellite positions.

However, implementing a successful core-satellite strategy requires solving a critical challenge: systematically identifying which satellite investments will generate positive alpha over the next holding period. Traditional approaches rely on fundamental analysis, sector rotation, or discretionary market timing, which are difficult to execute consistently and objectively.

This document presents a systematic, signal-based approach to satellite selection that addresses these challenges through rigorous quantitative methods. Rather than relying on subjective judgment or narrative-driven investing, we develop a comprehensive feature engineering framework that evaluates over 112,000 potential predictive signals derived from price dynamics, technical indicators, and cross-sectional relationships. Crucially, we apply statistical significance testing to distinguish genuine alpha sources from noise.

\subsection{Strategy Overview}

Our core-satellite strategy is designed for a 30+ year investment horizon with the following key characteristics:

\begin{itemize}
    \item \textbf{Core allocation (60\%):} iShares MSCI ACWI provides global diversification across approximately 3,000 stocks in developed and emerging markets, serving as both the portfolio foundation and the benchmark for measuring alpha.

    \item \textbf{Satellite allocation (40\%):} 5 dynamically selected ETFs from a universe of approximately 500 eligible candidates, chosen monthly based on predicted 1-month forward alpha using IC-weighted feature ensembles.

    \item \textbf{Monthly rebalancing:} Every month, we recalculate satellite selections and rebalance the portfolio to maintain the 60/40 ratio. This process locks profits from successful satellites into the core allocation while providing capital to new satellite positions.

    \item \textbf{Monthly contributions:} EUR 1,000 per month split 60/40 between core and satellites, implementing dollar-cost averaging to reduce timing risk.

    \item \textbf{Low transaction costs:} EUR 1--3 per trade with no capital gains taxes, making frequent rebalancing economically feasible.
\end{itemize}

The strategy's defining feature is the profit-locking mechanism: when satellites outperform, the monthly rebalancing automatically sells a portion of the gains to increase the core allocation, reducing portfolio risk over time. Conversely, when satellites underperform, capital from the stable core is deployed to satellite positions at lower valuations.

\subsection{Key Findings}

Through extensive walk-forward backtesting and statistical analysis, we arrive at several key findings:

\begin{enumerate}
    \item \textbf{IC weighting is statistically significant:} Weighting features by their Information Coefficient (correlation with forward alpha) produces a p-value of 0.047 in paired t-tests against the baseline, representing the only statistically significant improvement at the 5\% level.

    \item \textbf{N=5 satellites is optimal:} Analysis across N=1 to N=10 shows that N=5 provides the best balance of alpha generation and consistency.

    \item \textbf{Dynamic N selection adds risk without significance:} While dynamic adjustment of satellite count based on signal confidence can increase raw alpha, it also doubles maximum drawdown and fails to achieve statistical significance versus the simpler IC-weighted approach.

    \item \textbf{Other improvements fail significance tests:} Stability weighting, time-series features, regime-adaptive signals, and multi-factor interactions all fail to demonstrate statistically significant improvement after Bonferroni correction for multiple comparisons.

    \item \textbf{Results are robust across time:} Sub-period analysis confirms that outperformance is consistent across early, middle, and late portions of the backtest, ruling out early-luck compounding.
\end{enumerate}

\subsection{Contribution}

This document makes several contributions to systematic portfolio management:

\textbf{Comprehensive signal framework:} We design and implement 167 signal bases across 27 categories, capturing momentum, mean reversion, risk dynamics, trend following, and regime-dependent behaviors. Combined with 27 causal smoothing filters and 25 indicator transformations, this yields 112,725 features for evaluation.

\textbf{Rigorous causality:} All filters and transformations are strictly causal (using only past data), and we employ walk-forward backtesting to ensure performance estimates reflect realistic, implementable strategies without look-ahead bias.

\textbf{Statistical significance testing:} Rather than selecting strategies based solely on backtest performance, we apply paired t-tests, Wilcoxon signed-rank tests, and bootstrap confidence intervals to identify improvements that are statistically distinguishable from the baseline.

\textbf{Robustness verification:} We conduct sub-period analysis, year-by-year breakdowns, and rolling alpha analysis to verify that results are not driven by specific time periods or early luck compounding.

\textbf{Practical implementation:} The strategy is designed for real-world execution with realistic constraints including transaction costs, portfolio constraints (fixed 5 satellites), and tax considerations.

\subsection{Document Structure}

The remainder of this document is organized as follows: Section~\ref{sec:related_work} reviews relevant literature on core-satellite portfolios, factor investing, and technical analysis. Section~\ref{sec:methodology} details our signal construction, filtering, feature engineering, and selection methodology. Section~\ref{sec:results} presents backtesting results and statistical significance analysis. Section~\ref{sec:discussion} examines the robustness of our findings and discusses the implications of our statistical tests. Section~\ref{sec:conclusion} summarizes key insights and provides the final strategy recommendation.
