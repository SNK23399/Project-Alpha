\appendix
\section*{Appendix: Signal Framework Details}
\label{sec:appendix}

This appendix provides complete documentation of the signal framework used for ETF selection. The framework consists of three stages: signal base computation, smoothing filter application, and indicator transformation.

\subsection*{Signal Bases}

Tables~\ref{tab:signal-bases} and \ref{tab:signal-bases-2} list all 95 signal bases computed from raw price data. Each signal captures a different aspect of ETF behavior relative to the core benchmark (ACWI). Signals are organized into categories: returns and alpha, relative strength, risk metrics, risk-adjusted performance, trend indicators, mean reversion signals, higher moments, correlation, momentum dynamics, signal disagreement, regime-adaptive signals, and seasonality.

% Signal Bases Table - All signals with parameters in dedicated column
% Split into three tables for better fit (page-wide)

\begin{table*}[htbp]
\centering
\caption{Signal Bases (Part 1): Returns, Momentum, Risk \& Quality}
\label{tab:signal-bases}
\small
\begin{tabular}{@{}llp{7cm}@{}}
\toprule
\textbf{Signal} & \textbf{Windows (days)} & \textbf{Description} \\
\midrule
\multicolumn{3}{c}{\textit{Returns \& Alpha}} \\
\midrule
Daily Return & -- & $(P_t - P_{t-1})/P_{t-1}$ \\
Alpha vs Core & -- & $r_{ETF} - r_{Core}$ \\
Alpha vs Universe & -- & $r_{ETF} - \bar{r}_{univ}$ \\
Price Ratio vs Core & -- & $P_{ETF}/P_{Core}$ \\
Price Ratio vs Universe & -- & $r_{ETF}/\bar{r}_{univ}$ \\
Log Ratio vs Core & -- & $\ln(P_{ETF}/P_{Core})$ \\
Log Ratio vs Universe & -- & $\ln(r_{ETF}/\bar{r}_{univ})$ \\
Cumulative Return & 21, 63, 126, 252 & $\textstyle\sum(P_t/P_{t-1} - 1)$ \\
Cumulative Alpha vs Core & 21, 63, 126, 252 & $\textstyle\sum(r_{ETF} - r_{Core})$ \\
Cumulative Alpha vs Universe & 21, 63, 126, 252 & $\textstyle\sum(r_{ETF} - \bar{r}_{univ})$ \\
\midrule
\multicolumn{3}{c}{\textit{Momentum}} \\
\midrule
Momentum & 21, 63, 126, 252 & $(P_t - P_{t-n})/P_{t-n}$ \\
Relative Strength vs Core & 21, 63, 126, 252 & Mom$_{ETF}$/Mom$_{Core}$ \\
Relative Strength vs Universe & 21, 63, 126, 252 & Mom$_{ETF}$/Mom$_{univ}$ \\
Skip-Month Momentum & $t$: 63, 126, 252; $s$: 21, 42, 63 & $t$-day return, skip $s$ days \\
52-Week High Proximity & -- & $P_t / P_{52w,high}$ \\
52-Week Low Proximity & -- & $P_t / P_{52w,low}$ \\
Rate of Change (ROC) & 10, 20 & $(P_t - P_{t-n})/P_{t-n} \times 100$ \\
\midrule
\multicolumn{3}{c}{\textit{Risk Metrics}} \\
\midrule
Beta & 21, 63, 126, 252 & Cov/Var with core \\
Volatility & 21, 63, 126, 252 & $\sigma \times \sqrt{252}$ \\
Downside Deviation & 21, 63, 126, 252 & $\sqrt{\text{E}[\min(r,0)^2]} \times \sqrt{252}$ \\
Idiosyncratic Return & 21, 63, 126, 252 & $r_{ETF} - \beta_w \cdot r_{Core}$ (CAPM residual) \\
Relative Volatility & 21, 63, 126, 252 & $\sigma_{ETF}/\sigma_{Core}$ \\
Drawdown & -- & $(P - P_{max})/P_{max}$ \\
Relative Drawdown & -- & DD$_{ETF}$ - DD$_{Core}$ \\
Drawdown Duration & -- & Days since last peak \\
Recovery Rate & 21, 63, 126, 252 & $\Delta$DD / time underwater \\
CVaR (95\%) & 21, 63, 126, 252 & $\text{E}[r \mid r < \text{VaR}_{0.05}]$ \\
Ulcer Index & 21, 63, 126, 252 & $\sqrt{\text{mean}(\text{DD}^2)}$ \\
\midrule
\multicolumn{3}{c}{\textit{Risk-Adjusted Performance}} \\
\midrule
Sharpe Ratio & 21, 63, 126, 252 & $\bar{r}/\sigma \times \sqrt{252}$ \\
Information Ratio & 21, 63, 126, 252 & $\bar{\alpha}/\sigma_\alpha \times \sqrt{252}$ \\
Sortino Ratio & 21, 63, 126, 252 & $\bar{r}/\sigma_{down} \times \sqrt{252}$ \\
Calmar Ratio & 21, 63, 126, 252 & $\bar{r}/|\text{MaxDD}|$ \\
Treynor Ratio & 21, 63, 126, 252 & $(\bar{r} - r_f)/\beta$ \\
Omega Ratio & 21, 63, 126, 252 & $\sum(r > 0) / |\sum(r < 0)|$ \\
Gain-to-Pain Ratio & 21, 63, 126, 252 & $\sum r^+ / |\sum r^-|$ \\
Ulcer Performance Index & 21, 63, 126, 252 & $(\bar{r} - r_f)/\text{Ulcer}$ \\
Recovery Factor & 21, 63, 126, 252 & Total return / $|\text{MaxDD}|$ \\
\midrule
\multicolumn{3}{c}{\textit{Win/Loss Analysis}} \\
\midrule
Win Rate & 21, 63, 126, 252 & $\#(r > 0) / \#(r \neq 0)$ \\
Payoff Ratio & 21, 63, 126, 252 & $\bar{r}^+ / |\bar{r}^-|$ \\
Profit Factor & 21, 63, 126, 252 & $\sum r^+ / |\sum r^-|$ \\
Tail Ratio & 21, 63, 126, 252 & $|P_{95}| / |P_5|$ of returns \\
Stability of Returns & 21, 63, 126, 252 & $R^2$ of cumulative log returns \\
\midrule
\multicolumn{3}{c}{\textit{Beta-Adjusted Relative Strength}} \\
\midrule
Beta-Adjusted RS & $d$: 0.3, 0.5, 1.0; $\beta_w$: 21, 63, 126, 252 & RS$_{252}/(|\beta| + d)$ \\
\midrule
\multicolumn{3}{c}{\textit{Trend Indicators}} \\
\midrule
Price vs Moving Average & 20, 50, 100, 200 & $P/\text{MA}_n - 1$ \\
Moving Average Crossover & 20/50, 50/200 & MA$_s$/MA$_l$ - 1 \\
MACD & 12/26/9 & EMA$_{12}$ - EMA$_{26}$, signal = EMA$_9$ \\
PPO (Percentage Price Osc) & 12/26/9 & (EMA$_{12}$ - EMA$_{26}$)/EMA$_{26} \times 100$ \\
DPO (Detrended Price Osc) & 20 & $P - \text{MA}_{n/2+1 \text{ ago}}$ \\
TRIX & 15 & ROC of triple EMA \\
KST (Know Sure Thing) & 10/15/20/30 & Weighted sum of 4 ROC periods \\
Aroon Oscillator & 25 & Aroon$_{up}$ - Aroon$_{down}$ \\
\bottomrule
\end{tabular}
\end{table*}

\begin{table*}[htbp]
\centering
\caption{Signal Bases (Part 2): Mean Reversion, Dynamics \& Regime}
\label{tab:signal-bases-2}
\small
\begin{tabular}{@{}llp{7cm}@{}}
\toprule
\textbf{Signal} & \textbf{Windows (days)} & \textbf{Description} \\
\midrule
\multicolumn{3}{c}{\textit{Mean Reversion -- Z-Scores (inverted for buy signals)}} \\
\midrule
Price Z-Score & 21, 63, 126, 252 & $-(P - \bar{P})/\sigma$ \\
Alpha Z-Score & 21/252, 63/252, 126/252, 252/252 & $-(\sum \alpha_n) / (\sigma_{252} \times \sqrt{n})$ \\
RS Z-Score & 21, 63, 126, 252 & Inv. RS z-score \\
Distance from MA & 20, 50, 100, 200 & $-(P - \text{MA})/\text{MA}$ \\
Price Ratio Z-Score & 252 & Inv. price ratio z \\
\midrule
\multicolumn{3}{c}{\textit{Mean Reversion -- Technical Oscillators}} \\
\midrule
Bollinger Reversion & 20, 2$\sigma$ & $1 - (P - \text{lower})/(\text{upper} - \text{lower})$ \\
RSI Reversion & 14 & $50 - \text{RSI}$ \\
Stochastic Reversion & 14 & $50 - K$ (close-only) \\
Williams \%R & 14 & $(H_n - C)/(H_n - L_n) \times -100$ \\
TSI (True Strength Index) & 25/13 & Double-smoothed price change ratio \\
CCI (Commodity Channel Index) & 20 & $(P - \text{SMA})/(0.015 \times \text{MAD})$ \\
Ultimate Oscillator & 7/14/28 & Weighted 3-timeframe buying pressure \\
Donchian Position & 20 & $(P - L_n)/(H_n - L_n)$ \\
\midrule
\multicolumn{3}{c}{\textit{Mean Reversion -- Drawdown-Based}} \\
\midrule
Drawdown Reversion & -- & $-$drawdown \\
Alpha Drawdown Reversion & 21, 63, 126, 252 & $-(\sum \alpha_n - \max(\sum \alpha_n))$ \\
Sector Rotation Reversion & 21, 63, 126, 252 & $-$RS$_n$ (contrarian) \\
\midrule
\multicolumn{3}{c}{\textit{Higher Moments \& Complexity}} \\
\midrule
Skewness & 21, 63, 126 & Rolling skewness \\
Kurtosis & 21, 63, 126 & Rolling excess kurtosis \\
Return Autocorrelation & 21, 63, 126, 252 & Corr$(r_t, r_{t-1})$ over window \\
Hurst Exponent & 63, 126, 252 & R/S analysis trend persistence \\
Return Entropy & 63, 126 & $-\sum p_i \log p_i$ of return bins \\
\midrule
\multicolumn{3}{c}{\textit{Correlation \& Capture Ratios}} \\
\midrule
Core Correlation & 21, 63, 126, 252 & Corr$(r_{ETF}, r_{Core})$ \\
Diversification Benefit & 63 & $1 -$ core correlation \\
Market Correlation & 63 & Avg pairwise corr \\
Return Dispersion & 63 & Cross-sectional $\sigma$ of returns \\
Crowding/Herding & 63 & \% of ETFs with same momentum sign \\
Up Capture Ratio & 21, 63, 126, 252 & $\bar{r}_{ETF}^{up} / \bar{r}_{Core}^{up}$ when core $> 0$ \\
Down Capture Ratio & 21, 63, 126, 252 & $\bar{r}_{ETF}^{down} / \bar{r}_{Core}^{down}$ when core $< 0$ \\
Capture Ratio Spread & 21, 63, 126, 252 & Up Capture $-$ Down Capture \\
\midrule
\multicolumn{3}{c}{\textit{Momentum Dynamics}} \\
\midrule
Momentum Acceleration & $n$: 126, 252; $h$: 21, 63 & $h$-day $\Delta$ in RS$_n$ \\
\midrule
\multicolumn{3}{c}{\textit{Signal Disagreement}} \\
\midrule
Bullish Disagreement & 21, 252 & RS$_{252}$ - RS$_{21}$ \\
Bearish Disagreement & 21, 252 & RS$_{21}$ - RS$_{252}$ \\
Timeframe Disagreement & 21/63, 63/126, 126/252, 63/252 & RS$_l$ - RS$_s$ (adjacent pairs) \\
Trend-Momentum Divergence & 50, 63 & PriceMA$_z$ - RS$_z$ \\
\midrule
\multicolumn{3}{c}{\textit{Regime-Adaptive Signals}} \\
\midrule
Volatility Regime & 63/252 & Core vol z-score \\
Trend Regime & 200 & Core vs MA$_{200}$ \\
Drawdown Regime & -- & Core drawdown state \\
Dispersion Regime & 63 & Cross-sectional return spread \\
Trend-Boosted Momentum & 126, 200 & RS$_{126} \times$ trend \\
Volatility-Boosted Reversion & 63, 252 & AlphaDD $\times$ vol \\
\midrule
\multicolumn{3}{c}{\textit{Seasonality}} \\
\midrule
Month-of-Year Effect & -- & Binary: 1 if favorable month (Nov--Apr) \\
Month Sine/Cosine & -- & $\sin/\cos(2\pi \cdot \text{month}/12)$ \\
\bottomrule
\end{tabular}

\vspace{0.5em}
\small
\textbf{Note:} All window parameters use days: 21d $\approx$ 1 month, 63d $\approx$ 3 months, 126d $\approx$ 6 months, 252d $\approx$ 1 year. Total: \textbf{167 signal bases}. All signals validated against industry-standard libraries (ta, empyrical, quantstats).
\end{table*}


\subsection*{Smoothing Filters}

Table~\ref{tab:filters} describes the 14 causal smoothing filter types (24 configurations with different parameters) applied to each signal base. All filters are strictly causal---they only use past data at each time step, ensuring no look-ahead bias in the backtest. The filters range from simple exponential moving averages to adaptive filters (KAMA, regime-switching) and sophisticated signal processing methods (Kalman, Butterworth, Savitzky-Golay).

% Smoothing Filters Table - 14 filter types, 25 configurations applied to signal bases

\begin{table*}[htbp]
\centering
\caption{Causal Smoothing Filters Applied to Signal Bases}
\label{tab:filters}
\small
\begin{tabular}{@{}llp{7cm}@{}}
\toprule
\textbf{Filter} & \textbf{Parameters} & \textbf{Description} \\
\midrule
Raw & -- & Unfiltered signal $S_t$ passed through directly \\
Exponential MA & span: 21, 63 & $S_t^{ema} = \alpha S_t + (1-\alpha) S_{t-1}^{ema}$, $\alpha = 2/(n+1)$ \\
Double EMA & span: 21, 63 & $2 \cdot \text{EMA} - \text{EMA}(\text{EMA})$, reduced lag \\
Triple EMA & span: 21, 63 & $3 \cdot \text{EMA} - 3 \cdot \text{EMA}^2 + \text{EMA}^3$, minimal lag \\
Zero-Lag EMA & span: 21, 63 & EMA of $(2S_t - S_{t-\text{lag}})$, momentum-adjusted \\
Hull MA & period: 21, 63 & WMA$(2 \cdot \text{WMA}_{n/2} - \text{WMA}_n, \sqrt{n})$, low lag \\
Triangular MA & period: 21, 63 & SMA(SMA$(S_t)$), double-smoothed, center-weighted \\
Gaussian MA & window: 21, 63 & Bell-curve weights $w_i \propto e^{-i^2/2\sigma^2}$, smooth rolloff \\
Kaufman Adaptive & period: 21; fast/slow: 2/30 & Adapts $\alpha$ via efficiency ratio, trend-sensitive \\
Median Filter & window: 21, 63 & Rolling median, outlier-robust, preserves edges \\
Regime-Switching & fast/slow: 10/50; thresh & Uses fast MA in high-vol, slow MA in low-vol \\
Butterworth & cutoff: 21, 63; order=2 & Low-pass filter, attenuates $f > 1/n$ day$^{-1}$ \\
Kalman Filter & $Q$: $10^{-4}$, $10^{-5}$; $R$=$10^{-2}$ & Steady-state Kalman, adapts to signal noise \\
Savitzky-Golay & window: 21, 63; poly=3 & Cubic polynomial fit, preserves local peaks \\
\bottomrule
\end{tabular}

\vspace{0.5em}
\small
\textbf{Note:} All filters are strictly causal---they only use past data at each time step. 21 days $\approx$ 1 month, 63 days $\approx$ 3 months.
\end{table*}


\subsection*{Indicator Transformations}

Table~\ref{tab:indicators} documents the 25 indicator transformations applied to filtered signals. These indicators capture different temporal and cross-sectional aspects of signal behavior, including momentum at multiple horizons, z-scores, trend ratios, mean reversion, velocity, acceleration, and volatility ratios.

% Indicator Transformations Table - indicators computed from filtered signals

\begin{table*}[htbp]
\centering
\caption{Indicator Transformations Applied to Filtered Signals}
\label{tab:indicators}
\small
\begin{tabular}{@{}llp{7cm}@{}}
\toprule
\textbf{Indicator} & \textbf{Parameters} & \textbf{Description} \\
\midrule
\multicolumn{3}{c}{\textit{Level \& Momentum}} \\
\midrule
Level & -- & Raw filtered signal $S_t$ passed through directly \\
Momentum & horizon: 5, 21, 63, 126 & $(S_t - S_{t-h})/|S_{t-h}|$, rate of change \\
Momentum Acceleration & horizon: 5, 21, 63 & $\Delta(\text{mom}_h)$, is momentum speeding up? \\
Velocity & norm: 5, 21, 63 & $\Delta S_t / \sigma_w$, normalized first derivative \\
Acceleration & norm: 5, 21, 63 & $\Delta^2 S_t / \sigma_w$, normalized second derivative \\
Curvature & norm: 5, 21, 63 & $\Delta^2 S_t / |\Delta S_t|$, detects inflection points \\
\midrule
\multicolumn{3}{c}{\textit{Statistical Normalization}} \\
\midrule
Z-Score & window: 21, 63, 126, 252 & $(S_t - \bar{S}_w)/\sigma_w$, time-series standardization \\
Cross-Sectional Z-Score & -- & $(S_i - \bar{S}_{xs})/\sigma_{xs}$, relative to peers \\
Cross-Sectional Rank & -- & Sigmoid of cross-sectional z-score, bounded $[0,1]$ \\
Percentile & window: 63, 126, 252 & $(S_t - \bar{S}_w)/(2\sigma_w)$, historical position \\
\midrule
\multicolumn{3}{c}{\textit{Trend Indicators}} \\
\midrule
Trend Short & MA: 5/21 & $(\bar{S}_5 - \bar{S}_{21})/|\bar{S}_{21}|$, short-term trend \\
Trend Medium & MA: 21/63 & $(\bar{S}_{21} - \bar{S}_{63})/|\bar{S}_{63}|$, medium-term trend \\
Trend Long & MA: 63/126 & $(\bar{S}_{63} - \bar{S}_{126})/|\bar{S}_{126}|$, long-term trend \\
Trend Extended & MA: 126/252 & $(\bar{S}_{126} - \bar{S}_{252})/|\bar{S}_{252}|$, extended trend \\
Divergence & window: 21, 63 & $S_t - \bar{S}_w$, signal vs smoothed version \\
\midrule
\multicolumn{3}{c}{\textit{Mean Reversion}} \\
\midrule
Reversion & window: 21, 63, 126, 252 & $(S_t - \bar{S}_w)/|\bar{S}_w|$, deviation from mean \\
Envelope & window: 21, 63 & $(S_t - \bar{S}_w)/(k \cdot \sigma_w)$, Bollinger-style bands \\
\midrule
\multicolumn{3}{c}{\textit{Breakout \& Range}} \\
\midrule
Distance to High & window: 21, 63, 126 & $(S_t - S_{max})/\sigma_w$, breakout detection \\
Distance to Low & window: 21, 63, 126 & $(S_t - S_{min})/\sigma_w$, breakdown detection \\
Drawdown & window: 21, 63, 126 & $(S_t - S_{max})/|S_{max}|$, decline from peak \\
Range Position & window: 21, 63, 126 & $(S_t - S_{min})/(S_{max} - S_{min})$, position in range \\
Ratio to Peak & window: 21, 63, 126 & $S_t / S_{max}$, percentage of peak value \\
\midrule
\multicolumn{3}{c}{\textit{Volatility Metrics}} \\
\midrule
Volatility Ratio & short/long: 5/21, 21/63, 63/126 & $\sigma_{short}/\sigma_{long}$, volatility regime indicator \\
Relative Volatility & short/long: 21/252, 63/252 & $\sigma_{short}/\sigma_{long}$, short vs long-term vol \\
Signal-to-Noise & window: 21, 63, 126 & $|\bar{S}_w|/\sigma_w$, signal clarity measure \\
Roughness & window: 21, 63, 126 & $\sum|\Delta S|/|S_T - S_0|$, path noise vs net move \\
\midrule
\multicolumn{3}{c}{\textit{Higher Moments}} \\
\midrule
Skewness & window: 21, 63, 126 & Rolling asymmetry of signal distribution \\
Kurtosis & window: 21, 63, 126 & Rolling peakedness, extreme event likelihood \\
\midrule
\multicolumn{3}{c}{\textit{Regime Indicators}} \\
\midrule
Above Mean & window: 21, 63, 126 & Binary: $S_t > \bar{S}_w$, simple regime indicator \\
\midrule
\multicolumn{3}{c}{\textit{Cross-Sectional Dynamics}} \\
\midrule
Dispersion & -- & Cross-sectional $\sigma$ of signal, market disagreement \\
Convergence & window: 21, 63 & $\Delta\sigma_{xs}$, is signal converging across ETFs? \\
\bottomrule
\end{tabular}

\vspace{0.5em}
\small
\textbf{Note:} 95 signals $\times$ 25 filters $\times$ 76 indicators = \textbf{180,500 features}. All indicators are strictly causal.
\end{table*}


\subsection*{Feature Space Summary}

Combining all components yields the complete feature space:

\begin{itemize}
    \item \textbf{Signal Bases}: 95 fundamental signals
    \item \textbf{Smoothing Filters}: 24 causal filter configurations (including raw)
    \item \textbf{Indicators}: 25 transformations
    \item \textbf{Total Features}: $95 \times 24 \times 25 = 57{,}000$
\end{itemize}

Each feature is named using the convention: \texttt{signal\_\_filter\_\_indicator}, for example \texttt{vol\_boosted\_reversion\_\_hull\_63d\_\_xs\_rank}.

The GPU-accelerated computation pipeline processes all 57,000 features efficiently using:
\begin{itemize}
    \item CuPy for parallel computation on NVIDIA GPUs
    \item Streaming mode to limit memory usage
    \item Disk caching for rolling statistics to avoid redundant computation
    \item Vectorized operations for signal bases and indicators
\end{itemize}
