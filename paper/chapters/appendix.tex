\appendix
\section*{Appendix: Signal Framework Details}
\label{sec:appendix}

This appendix provides complete documentation of the signal framework used for ETF selection. The framework consists of three stages: signal base computation, smoothing filter application, and indicator transformation.

\subsection*{Signal Bases}

We compute 167 signal bases from raw price data, organized into 27 categories. Each signal captures a different aspect of ETF behavior relative to the core benchmark (ACWI). Signals use day-based window parameters (21d, 63d, 126d, 252d) corresponding approximately to 1, 3, 6, and 12 months respectively.

\textbf{Categories include:}
\begin{itemize}
    \item Returns and alpha (daily returns, excess returns, price ratios)
    \item Relative strength (multi-horizon momentum vs benchmark)
    \item Risk metrics (beta, volatility, drawdown)
    \item Risk-adjusted performance (Sharpe, Sortino, Information ratios)
    \item Trend indicators (price vs moving averages, golden cross)
    \item Mean reversion signals (z-scores, Bollinger bands, RSI, oversold indicators)
    \item Higher moments (skewness, kurtosis)
    \item Correlation and dispersion
    \item Momentum dynamics (acceleration, deceleration)
    \item Signal disagreement (short vs long-term momentum divergence)
    \item Regime-adaptive signals (volatility-boosted, trend-boosted)
\end{itemize}

\subsection*{Smoothing Filters}

We apply 27 causal smoothing filter configurations to each signal base. All filters are strictly causal---they only use past data at each time step, ensuring no look-ahead bias.

\textbf{Filter types:}
\begin{itemize}
    \item Raw (no filter) --- 1 configuration
    \item Exponential Moving Average (EMA) --- 21d, 63d
    \item Double EMA (DEMA) --- 21d, 63d
    \item Triple EMA (TEMA) --- 21d, 63d
    \item Zero-Lag EMA (ZLEMA) --- 21d, 63d
    \item Hull Moving Average --- 21d, 63d
    \item Triangular MA (TRIMA) --- 21d, 63d
    \item Gaussian MA --- 21d, 63d
    \item Butterworth Low-Pass --- 21d, 63d
    \item Kalman Filter --- fast, slow configurations
    \item Savitzky-Golay --- 21d, 63d
    \item Median Filter --- 21d, 63d
    \item Kaufman Adaptive MA (KAMA) --- 21d
    \item Regime-Switching Filter --- adaptive
\end{itemize}

\subsection*{Indicator Transformations}

Each filtered signal undergoes 25 indicator transformations that capture different aspects of signal behavior:

\begin{itemize}
    \item Level (raw filtered value)
    \item Momentum (5d, 21d, 63d, 126d changes)
    \item Z-scores (21d, 63d, 126d windows)
    \item Cross-sectional rank and z-score
    \item Trend ratios (MA ratios at multiple horizons)
    \item Mean reversion (distance from rolling means)
    \item Velocity and acceleration
    \item Drawdown (distance from rolling maximum)
    \item Range position (within high-low range)
    \item Volatility ratios (short vs long-term)
    \item Percentile (within rolling distribution)
    \item Relative volatility (vs benchmark)
\end{itemize}

\subsection*{Feature Space Summary}

Combining all components yields the complete feature space:

\begin{itemize}
    \item \textbf{Signal Bases}: 167 fundamental signals
    \item \textbf{Smoothing Filters}: 27 causal filter configurations
    \item \textbf{Indicators}: 25 transformations
    \item \textbf{Total Features}: $167 \times 27 \times 25 = 112{,}725$
\end{itemize}

Each feature is named using the convention: \texttt{signal\_\_filter\_\_indicator}, for example \texttt{rel\_strength\_126d\_\_hull\_63d\_\_xs\_rank}.

\subsection*{Computational Pipeline}

The signal computation pipeline processes all 112,725 features efficiently using:
\begin{itemize}
    \item Parallel processing with Python multiprocessing
    \item Vectorized numpy/pandas operations
    \item Parquet file caching for intermediate results
    \item SQLite database for signal storage
    \item Walk-forward architecture ensuring strict causality
\end{itemize}

\subsection*{Feature Selection for Ensemble}

From the 112,725 features, we select the top-performing features for the IC-weighted ensemble based on:
\begin{itemize}
    \item Historical average alpha generation
    \item IC Information Ratio $> 0.5$ (consistency)
    \item Hit rate $> 55\%$ (better than random)
    \item Low correlation with other selected features (diversification)
\end{itemize}

The final ensemble typically includes 50--100 features, weighted by their rolling 12-month IC values.
